\documentclass[12pt]{article}
\usepackage{amsmath, amssymb, amsthm,qcircuit}

\renewcommand{\>}{\rangle}
\newcommand{\<}{\langle}
\newcommand{\cL}{\mathcal{L}}
\newcommand{\cH}{\mathcal{H}}
\newcommand{\C}{\mathbb{C}}
\newcommand{\tr}{\mathrm{Tr}}

\setlength{\marginparwidth}{0pt} \setlength{\hoffset}{0cm}
\setlength{\oddsidemargin}{0pt} \setlength{\topmargin}{1cm}
\setlength{\headheight}{0pt} \setlength{\headsep}{0pt}
\addtolength{\textwidth}{3cm} \addtolength{\textheight}{5cm}

\begin{document}


\begin{center}
{\bf COT 5600 Quantum Computing} 

{\bf Spring 2019}

\bigskip

{\bf Homework 3}
\end{center}

\bigskip
Out: Wed 04/08

\bigskip
Due: Wed 04/19


\newpage

%%%%%%%%%%%%%%%%%%%%%%%%%%%%%%%%%%%%%%%%% PROBLEM 1

\noindent {\bf Problem 1} (Quantum Fourier transform)

\medskip
\noindent
Let $N=2^n$, $[N]=\{0,\ldots,N-1\}$, and $\omega=e^{2\pi i / N}$ be an $N$th root of unity. The Quantum Fourier transform $F_N$ of size $N$ is 
\[
F_N = \frac{1}{\sqrt{N}} \sum_{k,\ell\in[N]} \omega^{k\cdot \ell} |k\>\<\ell|\,.
\]
Show that $F_N$ is unitary.

\medskip
\noindent
To show that  $F_N$ is unitary, we have to show that $F^\dagger F = 1$ \\
$F = \frac{1}{\sqrt{N}} \sum\limits_{k,\ell \in[N]} e^\frac{2\pi i k \ell}{N} |k\>\<\ell| $ \\
$F^\dagger = \frac{1}{\sqrt{N}} \sum\limits_{k^*,\ell^* \in[N]} e^\frac{-2\pi i k^* \ell^*}{N} |\ell^*\>\<k^*| $ \\
$F^\dagger F = \frac{1}{\sqrt{N}}\frac{1}{\sqrt{N}} \sum\limits_{k,\ell, k^*,\ell^*\in[N]} e^\frac{2 \pi i (\ell^* k^* - \ell k)}{N} |\ell\>\<\ell^*| \delta_{k k^*}$ \\
$F^\dagger F = \frac{1}{N} \sum\limits{\ell k \ell^*} e^\frac{2 \pi i (\ell^* - \ell) k}{N}  |\ell\>\<\ell^*|$ \\
$F^\dagger F = \sum\limits_{\ell \ell^*}  |\ell\>\<\ell^*| \delta_{\ell \ell^*}$ \\
$F^\dagger F = I$


%%%%%%%%%%%%%%%%%%%%%%%%%%%%%%%%%%%%%%%%% PROBLEM 2

\newpage
\noindent {\bf Problem 2} (Quantum Phase estimation)

\medskip
\noindent
Let $\varphi\in[0,1)$ be arbitrary and
\[
|\varphi\> = \bigotimes_{k=n-1,\ldots,0} \frac{1}{\sqrt{2}} \left(  |0\> + \exp(2\pi i 2^k \varphi) |1\> \right) \,.
\]
Create a Python notebook that lets you compute and plot the probabilities for measuring $x\in\{0,1\}^n$ when the state is
\[
F_N^\dagger |\varphi\>
\]
for different $N$ and $\varphi$.  The plot should look similar to the plots on the slides depicting the different probability distributions.  Do not forget about the bit-reversal that we talked about in class.

\end{document}
